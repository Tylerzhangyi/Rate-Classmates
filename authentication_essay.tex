\documentclass[conference]{IEEEtran}
\IEEEoverridecommandlockouts
\onecolumn
% The preceding line is only needed to identify funding in the first footnote. If that is unneeded, please comment it out.
\usepackage{fontspec}
\usepackage{xeCJK}
% 设置中文字体 - 优先使用PingFang SC,否则使用Songti SC
\IfFontExistsTF{PingFang SC}{
    \setCJKmainfont{PingFang SC}
}{
    \setCJKmainfont{Songti SC}
}
% 设置英文正文字体
\setmainfont{Times New Roman}
% 设置代码字体
\setmonofont{Monaco}
\usepackage{cite}
\usepackage{amsmath,amssymb,amsfonts}
\usepackage{graphicx}
\usepackage{textcomp}
\usepackage{xcolor}
\usepackage{listings}
\usepackage{float}
\usepackage{tikz}
\usetikzlibrary{shapes,arrows,positioning}

% 代码高亮配置 - 专业蓝色主题
\definecolor{codegreen}{rgb}{0.13,0.55,0.13}      % 深绿色注释
\definecolor{codegray}{rgb}{0.5,0.5,0.5}          % 灰色行号
\definecolor{codepurple}{rgb}{0.58,0,0.82}        % 紫色字符串
\definecolor{codeblue}{rgb}{0.0,0.0,0.8}          % 蓝色关键字
\definecolor{backcolour}{rgb}{0.98,0.98,1.0}      % 浅蓝灰色背景

\lstdefinestyle{mystyle}{
    backgroundcolor=\color{backcolour},   
    commentstyle=\color{codegreen},
    keywordstyle=\color{codeblue}\bfseries,
    numberstyle=\tiny\color{codegray},
    stringstyle=\color{codepurple},
    basicstyle=\ttfamily\footnotesize,
    breakatwhitespace=false,         
    breaklines=true,                 
    captionpos=b,                    
    keepspaces=true,                 
    numbers=left,                    
    numbersep=5pt,                  
    showspaces=false,                
    showstringspaces=false,
    showtabs=false,                  
    tabsize=2,
    frame=single,
    frameround=tttt,
    rulecolor=\color{blue!30}
}

\lstset{style=mystyle}

% 调整间距
\setlength{\parskip}{0.4em}
\lstset{aboveskip=12pt, belowskip=12pt}
\setlength{\floatsep}{15pt plus 2pt minus 2pt}
\setlength{\textfloatsep}{25pt plus 2pt minus 4pt}
\setlength{\intextsep}{15pt plus 2pt minus 2pt}
\setlength{\abovecaptionskip}{10pt}
\setlength{\belowcaptionskip}{10pt}

\def\BibTeX{{\rm B\kern-.05em{\sc i\kern-.025em b}\kern-.08em
    T\kern-.1667em\lower.7ex\hbox{E}\kern-.125emX}}

\begin{document}

\title{基于Django Session机制的用户登录与授权系统实现}

\author{\IEEEauthorblockN{1\textsuperscript{st} 张翼}
\IEEEauthorblockA{\textit{计算机科学与技术系} \\
\textit{杭州云谷学校}\\
杭州, 中国 \\
tyler.zhangyi@yungu.org}
}

\maketitle

\begin{abstract}
本文阐述基于Django Session机制的用户登录与授权系统实现,回答四个核心问题:Cookie在浏览器中的工作原理及自动认证机制、服务器如何识别用户并发放Cookie、Session存储与用户关联方式、密码验证流程。
\end{abstract}

\begin{IEEEkeywords}
用户认证, Session管理, Cookie机制, Django框架, Web安全
\end{IEEEkeywords}

\section{引言}

本文基于Vue.js前端和Django后端的Web应用系统,阐述Django Session认证机制的实现,回答以下四个问题:(1) Cookie在浏览器中的工作原理及多请求自动认证;(2) 服务器如何识别用户并发放Cookie;(3) Session存储与用户关联机制;(4) 密码验证流程。

\section{Cookie机制与自动认证}

\subsection{Cookie在浏览器中的工作原理}

HTTP Cookie存储在浏览器中,服务器通过Set-Cookie响应头设置Cookie后,浏览器会在后续请求中自动在请求头中包含该Cookie。前端使用Axios库,关键配置如下:

\vspace{0.2cm}
\begin{lstlisting}[caption=前端API客户端配置]
import axios from 'axios'

const client = axios.create({
  baseURL: import.meta.env.VITE_API_BASE_URL || 
           'http://localhost:8000/api/',
  withCredentials: true  // 启用 cookie,用于 session 认证
})

export default client
\end{lstlisting}

\texttt{withCredentials: true} 使浏览器在跨域请求中自动包含Cookie。

\subsection{多请求场景下的自动认证}

登录成功后,服务器通过Set-Cookie设置\texttt{sessionid} Cookie。此后,浏览器在每次AJAX请求中自动在请求头包含该Cookie,无需前端代码处理。前端登录实现:

\vspace{0.2cm}
\begin{lstlisting}[caption=前端登录逻辑]
async function login(account, password) {
  loading.value = true
  error.value = ''
  try {
    const res = await client.post('auth/login', { 
      account, password 
    })
    const user = res.data.data
    currentUser.value = { ...user, name: user.account, 
                         id: user.user_id }
    localStorage.setItem('currentUser', 
                        JSON.stringify(currentUser.value))
    return { success: true, message: res.data.message }
  } catch (e) {
    error.value = e.message
    return { success: false, message: e.message }
  } finally {
    loading.value = false
  }
}
\end{lstlisting}

\texttt{localStorage}仅用于前端状态管理,认证依赖浏览器自动发送的Cookie。

服务器端Cookie配置:

\vspace{0.2cm}
\begin{lstlisting}[language=Python, caption=Session Cookie配置]
SESSION_COOKIE_NAME = "sessionid"
SESSION_COOKIE_AGE = 86400 * 7  # 7天过期
SESSION_COOKIE_HTTPONLY = True
SESSION_COOKIE_SAMESITE = "Lax"
SESSION_COOKIE_SECURE = False  # 开发环境
\end{lstlisting}

\texttt{HttpOnly=True}防止JavaScript访问,\texttt{SameSite=Lax}限制跨站发送,\texttt{Secure}在生产环境应设为True。

\section{服务器端用户识别与会话创建}

\subsection{登录流程概述}

登录流程如图\ref{fig:login_flow}所示。用户提交登录表单后,前端通过POST请求将账号和密码发送到\texttt{/api/auth/login}端点。

\vspace{0.2cm}
\begin{figure}[H]
\centering
\resizebox{0.4\textwidth}{!}{
\begin{tikzpicture}[node distance=0.5cm, auto, >=stealth, every node/.style={align=center, font=\scriptsize}]
    \node [draw, rectangle, minimum width=1.5cm, minimum height=0.4cm] (user) {用户输入\\账号密码};
    \node [draw, rectangle, below=of user, minimum width=1.5cm, minimum height=0.4cm] (frontend) {前端Vue\\组件};
    \node [draw, rectangle, below=of frontend, minimum width=1.5cm, minimum height=0.4cm] (api) {Django API\\客户端};
    \node [draw, rectangle, below=of api, minimum width=1.5cm, minimum height=0.4cm] (backend) {后端登录\\视图};
    \node [draw, rectangle, below=of backend, minimum width=1.5cm, minimum height=0.4cm] (db) {数据库\\验证};
    \node [draw, rectangle, below=of db, minimum width=1.5cm, minimum height=0.4cm] (session) {创建\\Session};
    \node [draw, rectangle, below=of session, minimum width=1.5cm, minimum height=0.4cm] (cookie) {返回\\Set-Cookie};
    
    \draw [->, thick] (user) -- (frontend);
    \draw [->, thick] (frontend) -- node[right, font=\tiny] {POST /auth/login} (api);
    \draw [->, thick] (api) -- node[right, font=\tiny] {HTTP请求} (backend);
    \draw [->, thick] (backend) -- node[right, font=\tiny] {查询用户} (db);
    \draw [->, thick] (db) -- node[right, font=\tiny] {验证成功} (session);
    \draw [->, thick] (session) -- node[right, font=\tiny] {sessionid} (cookie);
\end{tikzpicture}
}
\caption{用户登录流程图}
\label{fig:login_flow}
\end{figure}

\subsection{服务器端登录实现}

服务器端登录视图实现:

\vspace{0.2cm}
\begin{lstlisting}[language=Python, caption=服务器端登录视图]
@csrf_exempt
@allow_methods(["POST"])
def login_view(request: HttpRequest):
    payload = parse_json(request)
    account = payload.get("account")
    password = payload.get("password")
    try:
        user = User.objects.get(account=account, 
                                password=password)
    except User.DoesNotExist:
        return api_response(401, "账号或密码错误", 
                           status=401)
    # 将用户信息存入 session
    request.session['user_id'] = str(user.user_id)
    request.session['account'] = user.account
    request.session['role'] = user.role
    request.session.set_expiry(86400 * 7)  # 7天过期
    # 确保 session 被保存
    request.session.save()
    return api_response(200, "登录成功", 
                       _user_payload(user))
\end{lstlisting}

\vspace{0.3cm}
\subsection{Session创建与Cookie发放}

Django Session中间件处理流程:(1) 从请求Cookie读取\texttt{sessionid};(2) 加载Session数据到\texttt{request.session};(3) 视图函数读写Session;(4) 响应时,如果Session被修改,中间件生成sessionid、存储到\texttt{django\_session}表、在响应头设置\texttt{Set-Cookie: sessionid=<session\_key>}。\texttt{request.session.save()}确保Session持久化并触发Cookie生成。

\section{Session存储与用户关联}

\subsection{Session数据存储机制}

Django将Session数据存储在\texttt{django\_session}表:\texttt{session\_key}(主键,即sessionid)、\texttt{session\_data}(加密后的Session数据)、\texttt{expire\_date}(过期时间)。Session数据采用base64编码的pickle格式,经HMAC签名确保完整性。

\vspace{0.2cm}
\subsection{用户身份关联过程}

当用户后续请求到达服务器时,认证流程如下:

\vspace{0.2cm}
\begin{lstlisting}[language=Python, caption=认证装饰器]
def require_auth(view_func: Callable) -> Callable:
    """
    认证装饰器:检查用户是否已登录(通过 session)
    """
    @wraps(view_func)
    def wrapper(request: HttpRequest, *args, **kwargs):
        if 'user_id' not in request.session:
            return api_response(401, "未登录,请先登录", 
                               status=401)
        return view_func(request, *args, **kwargs)
    return wrapper
\end{lstlisting}

认证流程:(1) 浏览器在请求头自动包含Cookie: \texttt{sessionid=<session\_key>};(2) Session中间件读取sessionid,查询\texttt{django\_session}表;(3) 加载Session数据到\texttt{request.session};(4) 视图函数通过\texttt{request.session['user\_id']}获取用户ID;(5) 装饰器检查\texttt{user\_id},不存在则返回401。

\vspace{0.3cm}
\subsection{Session生命周期管理}

Session有效期通过\texttt{request.session.set\_expiry(86400 * 7)}设置为7天。登出时调用\texttt{request.session.flush()}删除Session记录:

\vspace{0.2cm}
\begin{lstlisting}[language=Python, caption=登出实现]
def logout_view(request: HttpRequest):
    """登出:清除 session"""
    request.session.flush()
    return api_response(200, "登出成功")
\end{lstlisting}

\section{密码验证机制}

\subsection{密码存储策略}

系统采用明文密码存储。用户模型定义:

\vspace{0.2cm}
\begin{lstlisting}[language=Python, caption=用户模型]
class User(models.Model):
    class Role(models.TextChoices):
        USER = "user", "user"
        ADMIN = "admin", "admin"
    
    user_id = models.UUIDField(primary_key=True, 
                               default=uuid.uuid4)
    account = models.CharField(max_length=255, unique=True)
    password = models.CharField(max_length=255)  
    # 按需求存明文
    role = models.CharField(max_length=10, 
                           choices=Role.choices, 
                           default=Role.USER)
    created_at = models.DateTimeField(auto_now_add=True)
\end{lstlisting}

\subsection{密码验证过程}

密码验证逻辑:

\vspace{0.2cm}
\begin{lstlisting}[language=Python, caption=密码验证逻辑]
try:
    user = User.objects.get(account=account, 
                            password=password)
except User.DoesNotExist:
    return api_response(401, "账号或密码错误", 
                       status=401)
\end{lstlisting}

验证步骤:(1) 接收\texttt{account}和\texttt{password};(2) 使用\texttt{User.objects.get(account=account, password=password)}查询;(3) 存在匹配记录则验证成功;(4) 抛出\texttt{User.DoesNotExist}异常则返回401。返回统一错误信息"账号或密码错误"可防止账号枚举攻击。

\section{完整认证流程}

图\ref{fig:complete_auth}展示了从登录到后续API请求的完整认证流程。

\begin{figure}[H]
\centering
\resizebox{0.9\textwidth}{!}{
\begin{tikzpicture}[node distance=3cm and 3.2cm, auto, >=stealth, every node/.style={align=center, font=\small}]
    % 登录流程 - 第一行
    \node [draw, rectangle, minimum width=2.5cm, minimum height=0.9cm] (login) {用户登录};
    \node [draw, rectangle, right=of login, minimum width=3cm, minimum height=0.9cm] (req1) {POST\\/auth/login};
    \node [draw, rectangle, right=of req1, minimum width=2.5cm, minimum height=0.9cm] (verify) {验证密码};
    
    % 登录流程 - 第二行
    \node [draw, rectangle, below=of verify, minimum width=3cm, minimum height=0.9cm] (session) {创建Session};
    \node [draw, rectangle, left=of session, minimum width=2.5cm, minimum height=0.9cm] (cookie) {Set-Cookie};
    \node [draw, rectangle, below=of cookie, minimum width=3cm, minimum height=0.9cm] (store) {存储Session\\到DB};
    
    % 后续请求流程 - 第一行
    \node [draw, rectangle, below=of login, yshift=-1.5cm, minimum width=3cm, minimum height=0.9cm] (api) {后续API\\请求};
    \node [draw, rectangle, right=of api, minimum width=3cm, minimum height=0.9cm] (req2) {GET/POST\\请求};
    \node [draw, rectangle, right=of req2, minimum width=2.5cm, minimum height=0.9cm] (check) {检查Session};
    
    % 后续请求流程 - 第二行
    \node [draw, rectangle, below=of check, minimum width=2.5cm, minimum height=0.9cm] (auth) {验证\\user\_id};
    \node [draw, rectangle, left=of auth, minimum width=2.5cm, minimum height=0.9cm] (allow) {允许访问};
    
    % 登录流程箭头
    \draw [->, thick] (login) -- (req1);
    \draw [->, thick] (req1) -- (verify);
    \draw [->, thick] (verify) -- (session);
    \draw [->, thick] (session) -- (cookie);
    \draw [->, thick] (cookie) -- (store);
    
    % 后续请求流程箭头
    \draw [->, thick] (login) -- node[left, xshift=-0.6cm, font=\tiny] {登录后} (api);
    \draw [->, thick] (api) -- node[above, font=\tiny] {Cookie\\自动发送} (req2);
    \draw [->, thick] (req2) -- (check);
    \draw [->, thick] (check) -- (auth);
    \draw [->, thick] (auth) -- (allow);
    
    % Session到后续请求的连接(使用更平滑的曲线)
    \draw [->, dashed, thick] (store) to[out=0, in=200] node[right, xshift=0.6cm, yshift=-0.3cm, font=\tiny] {Session\\关联} (req2);
\end{tikzpicture}
}
\caption{完整认证流程示意图}
\label{fig:complete_auth}
\end{figure}

\section{结论}

本文阐述了基于Django Session机制的用户认证系统实现,回答了四个核心问题:

\begin{enumerate}
    \item \textbf{Cookie自动发送}:配置\texttt{withCredentials: true}后,浏览器在所有AJAX请求中自动包含Session Cookie。
    
    \item \textbf{Session创建流程}:登录成功后,服务器将用户信息存入Session,Django中间件自动生成sessionid并通过Set-Cookie发送给浏览器。
    
    \item \textbf{用户身份关联}:Session数据存储在\texttt{django\_session}表,通过sessionid关联。每次请求时,Session中间件自动加载Session数据到\texttt{request.session}。
    
    \item \textbf{密码验证}:使用\texttt{User.objects.get(account=account, password=password)}直接比较验证。
\end{enumerate}

\begin{thebibliography}{00}
\bibitem{django_sessions}
Django Documentation. ``Sessions.'' [Online]. Available: https://docs.djangoproject.com/en/stable/topics/http/sessions/

\bibitem{http_cookies}
IETF RFC 6265. ``HTTP State Management Mechanism.'' [Online]. Available: https://tools.ietf.org/html/rfc6265

\bibitem{cors_spec}
Mozilla Developer Network. ``Cross-Origin Resource Sharing (CORS).'' [Online]. Available: https://developer.mozilla.org/en-US/docs/Web/HTTP/CORS

\bibitem{axios_docs}
Axios Documentation. ``Request Config.'' [Online]. Available: https://axios-http.com/docs/req\_config
\end{thebibliography}

\end{document}
